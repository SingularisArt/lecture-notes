\lesson{1}{Apr 04 2022 Mon (10:01:23)}{Sets and Numbers}
\label{les_1:sets_and_numbers}

\begin{definition}[Set]
  \label{def:set}

  A set is a collection of objects specified in a manner that enables one to
  determine if a given object is or isn't in the set.
\end{definition}

\begin{exc}[Solution \ref{sol:which_of_the_following_represent_a_set}]
  \label{exc:which_of_the_following_represent_a_set}

  Which of the following represent a set?
  
  \begin{enumerate}
    \label{enum:sets_and_numbers_example}

    \item The students registered for MTH $112$ at PCC this quarter.
    \item The good students registered for MTH $112$ at PCC this quarter.
  \end{enumerate}
\end{exc}

\begin{notation}
  Roster Notation involves listing the elements in a set within curly brackets:
  "$\{ \}$" like the following: "$\{ 1, 2, 3, 4 \}$"
\end{notation}

\begin{definition}[Element]
  \label{def:element}

  An object in a set is called an \textbf{element} of the set. (symbol: "$\in$")
\end{definition}

\begin{example}
  \label{exm:element}

  $5$ is an element of the set $\{ 4, 5, 6, 7, 8, 9 \}$. We can express this
  symbolically:

  \[ 5 \in \{ 4, 5, 6, 7, 8, 9 \} \].
\end{example}

\begin{definition}[Subset]
  \label{def:subset}

  A set $S$ of a set $T$, denoted $S \subseteq T$, if all elements of $S$ are
  also elements of $T$.

  If $S$ and $T$ are sets and $S = T$, then $S \subseteq T$. Sometimes it's
  useful to consider a subset $S$ of a set $T$ that isn't equal to $T$. In such
  case, we write $S \subset T$ and say that $S$ is a proper subset of $T$.
\end{definition}

\begin{example}
  \label{exm:subset}

  $\{ 4, 7, 8 \}$ is a subset of the set $\{ 4, 5, 6, 7, 8, 9 \}$.

  We can express this fact symbolically by
  $\{ 4, 7, 8 \} \subseteq \{ 4, 5, 6, 7, 8, 9 \}$ 

  Since these two sets aren't equal, $\{ 4, 7, 8 \}$ is a proper subset of
  $\{ 4, 5, 6, 7, 8, 9 \}$, so can write:
  \[ \{ 4, 7, 8 \} \subset \{ 4, 5, 6, 7, 8, 9 \} \].

  We can use the other symbol as follows:
  \[ \{ 1, 2, 3 \} \subseteq \{ 1, 2, 3 \} \].
\end{example}

\begin{definition}[Empty Set]
  \label{def:empty_set}

  The empty set, denoted $\emptyset$, is the set with no elements

  \[ \emptyset = \{ \} \].
\end{definition}

\begin{definition}[Union]
  \label{def:union}

  The union of two sets $A$ and $B$, denoted $A \cup B$, is the set containing
  all of the elements in either $A$ or $B$ (or both $A$ and $B$).
\end{definition}

\begin{example}
  \label{exm:union}

  Consider the sets $\{ 4, 7, 8 \}$, $\{ 0, 2, 4, 6, 8 \}$, and
  $\{ 1, 3, 5, 7 \}$. Then \ldots

  \begin{itemize}
    \label{item:union}
  
    \item $\{ 4, 7, 8 \} \cup  \{ 1, 3, 5, 7 \} = \{ 1, 3, 4, 5, 7, 8 \}$
    \item $\{ 4, 7, 8 \} \cup  \{ 0, 2, 4, 6, 8 \} = \{ 0, 2, 4, 6, 7, 8 \}$
    \item $\{ 0, 2, 4, 6, 8 \} \cup  \{ 1, 3, 5, 7 \} = \{ 0, 1, 2, 3, 4, 5, 6,
      7, 8 \}$
  \end{itemize}
\end{example}

\begin{definition}[Intersection]
  \label{def:intersection}

  The intersection of two sets $A$ and $B$, denoted $A \cap B$, is the set
  containing all the elements in both $A$ and $B$.
\end{definition}

\begin{example}
  \label{exm:intersection}

  Consider the sets $\{ 4, 7, 8 \}$, $\{ 0, 2, 4, 6, 8 \}$, and
  $\{ 1, 3, 5, 7 \}$. Then \ldots

  \begin{itemize}
    \label{item:intersection_1}
  
    \item $\{ 4, 7, 8 \} \cap \{ 0, 2, 4, 6, 8 \} = \{ 4, 8 \}$
    \item $\{ 4, 7, 8 \} \cap \{ 1, 3, 5, 7\} = \{ 7 \}$ 
    \item $\{ 0, 2, 4, 6, 8 \} \cap \{ 1, 3, 5, 7 \} = \emptyset$
  \end{itemize}
\end{example}

\begin{notation}
  Set Builder Notation.

  "All the whole numbers between $3$ and $10$" $= \{ x | x \in \mathbb{Z}$ and
  $3 < x < 10 \}$
\end{notation}

\begin{definition}[Important Sets of Numbers]
  \label{def:important_sets_of_numbers}

  The set of natural numbers:
  \[ \mathbb{N} = \{ 1, 2, 3, 4, 5, \ldots \} \].

  The set of integers:
  \[ \mathbb{Z} = \{ \ldots, -3, -2, -1, 0, 1, 2, 3, \ldots \} \].

  The set of rational numbers:
  \[ \mathbb{Q} = \left\{x | x = \frac{p}{q} \textrm{ and } p, q \in \mathbb{Z}
    \textrm{ and } q \ne 0 \right\} \].

  The set of real numbers: $\mathbb{R}$
  \[ \textrm{(All the numbers on the number line)} \].

  The set of complex numbers:
  \[ \mathbb{C} = \left\{x | x = a + bi \textrm{ and } a, b \in \mathbb{R}
    \textrm{ and } i = \sqrt{-1}\right\} \].
\end{definition}

\begin{note}
  $\mathbb{N} \subset \mathbb{Z} \subset \mathbb{Q} \subset \mathbb{R} \subset
  \mathbb{C}$, the set of natural numbers ($\mathbb{N}$) is a subset of the set
  of integers ($\mathbb{Z}$) which is a subset of the set of rational numbers
  ($\mathbb{Q}$) which is a subset of the set of real numbers ($\mathbb{R}$)
  which is a subset of the set of complex numbers ($\mathbb{C}$).
\end{note}

\begin{notation}
  Since we use the real numbers so often, we have special notation for subsets
  of the real numbers. \textbf{Interval Notation}. Interval Notation involves
  square or round brackets.
\end{notation}

\begin{example}
  \label{exm:interval_notation}

  Quick demo of Interval Notation

  \begin{itemize}
    \label{item:interval_notation}

    \item $\left\{x | x \in \mathbb{R} \textrm{ and } -2 \le x \le 3\right\}
      = [-2, 3]$
    \item $\left\{x | x \in \mathbb{R} \textrm{ and } -2 < x < 3\right\}
      = (-2, 3)$
    \item $\left\{x | x \in \mathbb{R} \textrm{ and } -2 < x \le 3\right\}
      = (-2, 3]$
    \item $\left\{x | x \in \mathbb{R} \textrm{ and } -2 \le x < 3\right\}
      = [-2, 3)$
  \end{itemize}

  When the interval has no upper or lower bound, we use the infinity symbol
  ($\infty$ or $-\infty$)

  \begin{itemize}
    \label{item:interval_notation_with_no_upper_or_lower_bound}

    \item $\left\{x | x \in \mathbb{R} \textrm{ and } x \le 4\right\} =
      (-\infty, 4]$
    \item $\left\{x | x \in \mathbb{R} \textrm{ and } x \ge 4\right\} =
      [4, -\infty)$
  \end{itemize}
\end{example}

\begin{exc}[Solution \ref{sol:simplify_the_following_expressions}]
  \label{exc:simplify_the_following_expressions}

  Simplify the following expressions:
  
  \begin{itemize}
    \label{item:simplify_the_following_expressions}

    \item $(-4, \infty) \cup [-8, 3]$
    \item $(-4, \infty) \cup (-\infty, 2]$
    \item $(-4, \infty) \cap (-\infty, 2]$
    \item $(-4, \infty) \cap [-10, -5]$
  \end{itemize}
\end{exc}

\newpage
