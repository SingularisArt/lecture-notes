\lesson{5}{Apr 16 2022 Sat (17:10:40)}{Part 2: Intro to Trig Functions}
\label{les_5:part_2_intro_to_trig_functions}

There are four other trigonometric functions besides sin and cos. These four
functions are defined in terms of sin and cos functions:

\begin{definition}[The Other $4$ Trig Functions]
  \label{def:the_other_4_trig_functions}

  The \textbf{tangent function}, denoted $\tan (\theta)$, is defined by:
  \[ \tan (\theta) = \frac{\sin (\theta)}{\cos (\theta)} \].

  The \textbf{cotangent function}, denoted $\cot (\theta)$, is defined by:
  \[ \cot (\theta) = \frac{1}{\tan (\theta)} = \frac{\cos (\theta)}{\sin (\theta)} \].

  The \textbf{secant function}, denoted $\sec (\theta)$, is defined by:
  \[ \sec (\theta) = \frac{1}{\cos (\theta)} \]. 

  The \textbf{cosecant function}, denoted $\csc (\theta)$, is defined by:
  \[ \csc (\theta) = \frac{1}{\sin (\theta)} \].
\end{definition}

The tangent function is the ratio of the sin and cos functions so it provides
truly different information than does sin or cos. So the tan function is
important. But the cot, sec, and csc are just reciprocals of tan, cos, and sin.
So they don't provide any new information, so they are arguably less important
functions.

There are $2$ other identities that can be obtained from the Pythagorean
Identity.

One of these identities can be found by dividing both sides of the Pythagorean
Identity by $\cos^{2} (\theta)$:

\begin{align*}
  &\qquad\sin^{2} (\theta) + \cos^{2} (\theta) = 1 \\
  &\implies \frac{\sin^{2} (\theta)}{\cos^{2} (\theta)} + \frac{\cos^{2} (\theta)}{\cos^{2} (\theta)} = \frac{1}{\cos^{2} (\theta)} \\
  &\implies \qquad \tan^{2} (\theta) + 1 = \sec^{2} (\theta)
.\end{align*}

Alternatively, we can divide both sides of the Pythagorean Identity by $\sin^{2}
(\theta)$ and find another identity.

\begin{align*}
  &\qquad\sin^{2} (\theta) + \cos^{2} (\theta) = 1 \\
  &\implies \frac{\sin^{2} (\theta)}{\sin^{2} (\theta)} + \frac{\cos^{2}
  (\theta)}{\sin^{2} (\theta)} = \frac{1}{\sin^{2} (\theta)} \\
  &\implies \qquad 1 + \cot^{2} (\theta) = \csc^{2} (\theta)
.\end{align*}

This gives us three identities that are considered \textbf{"The Pythagorean
Identities"}:

\begin{tcolorbox}
  \begin{center}
    The Pythagorean Identities
  \end{center}
  \begin{align*}
    &\sin^{2} (\theta) + \cos^{2} (\theta) = 1 \\
    &\tan^{2} (\theta) + 1 = \sec^{2} (\theta) \\
    &1 + \cot^{2} (\theta) = \csc^{2} (\theta)
  .\end{align*}
\end{tcolorbox}

\newpage
