\lesson{10}{Apr 30 2022 Sat (20:45:24)}{Solving Trig Equations}
\label{les_10:solving_trig_equations}

The inverse functions we constructed in lecture $9$ can be used to solve
equations like $\sin(t) = \frac{1}{2}$ but the fraction $\frac{1}{2}$ is a
"friendly" sin value so we don't need to use the inverse sin function. We know
that $\sin (\frac{\pi}{6}) = \frac{1}{2}$, so we know that $t = \frac{\pi}{6}$
is a solution to $\sin(t) = \frac{1}{2}$. We also know that the sin function is
periodic with period $2\pi$, so its values repeat every $2\pi$ units. We can
represent multiples of the period with the expression $2k\pi$, where $k$ is any
integer, $k \in \mathbb{Z}$, so we can represent all the solutions that
are "related" to $\frac{\pi}{6}$ with the expression $\frac{\pi}{6} + 2k\pi, k
\in \mathbb{Z}$. This expression represents infinitely many solutions, but
it still doesn't represent all the solutions.

Recall the identity $\sin (t) = \sin (\pi - t)$. This identity tells us that
angles $t$ and $\pi - t$ always have the same sin value. This means that
whenever we've found a solution, $t$, to an equation involving sin, we can find
another solution by computing $\pi - t$. Now let's apply this observation to
find the rest of the solutions to $\sin(t) = \frac{1}{2}$. Since we know that
$t = \frac{\pi}{6}$ is a solution we know that $t = \pi - \frac{\pi}{6} =
\frac{5\pi}{6}$ is another solution. And now we can again utilize the fact that
the period of the sin function is $2\pi$ so we can express the rest of the
solutions with $t = \frac{5\pi}{6} + 2k\pi, k \in \mathbb{Z}$. So the complete
solution to the equation $\sin (t) = \frac{1}{2}$ is:
\[
t = \frac{\pi}{6} + 2k\pi \textrm{ or } t = \frac{5\pi}{6} + 2k\pi \textrm{ for
all } k \in \mathbb{Z}
\].

\begin{exc}[Solution \ref{sol:inv_sin_2}]
  \label{exc:inv_sin_2}

  Solve the equation $\sin(t) = -0.555$.
\end{exc}

\newpage
